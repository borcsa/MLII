\section*{Exercise 2: Kernalized CCA}

\subsection*{Part (a)}

If we denote by $\Phi_1$ the map that maps the $X$-data into feature space and by $\Phi_2$ the map that maps the $Y$-data into (a possibly different) feature space. 
As in exercise 1 any solution can be found as a span of the data, so we set 
$w_x = \sum_{i=1}^{N}\alpha_i \Phi_1(x_i)$ and $w_y = \sum_{i=1}^{N}\beta_i \Phi_2(y_i)$.

The kernels are the inner products of data points in the feature space.
If we put this into the Problem and proceed as exercise 1 in the dual CCA we get nearly the same generalized eigenvalue problem, only with the Kernel matrizes instead of the normal ones, i.e. 
\begin{align*}
\begin{pmatrix}
0 & K_XK_Y \\
K_YK_X & 0
\end{pmatrix} \begin{pmatrix}
\alpha_x \\ \alpha_y
\end{pmatrix}
= \rho 
\begin{pmatrix}
K_X^2 & 0 \\
0 & K_Y^2 
\end{pmatrix}
\begin{pmatrix}
\alpha_x \\ \alpha_y
\end{pmatrix}
\end{align*}